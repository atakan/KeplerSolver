\documentclass{article}
\usepackage{graphicx}
\usepackage[fleqn]{amsmath}
\usepackage{natbib}
\usepackage{url}
\usepackage{hyperref}
\usepackage{ucs}
\usepackage[utf8x]{inputenc}
\bibliographystyle{plainnat.bst}

\usepackage[T1]{fontenc}
\usepackage{ae,aecompl}
\usepackage{textcomp}

\newcommand{\swift}{{\tt SWIFT}}

\author{M. Atakan Gürkan}
\title{Solving Kepler's Equation}
\date{}
\begin{document}
\maketitle
\section{The problem}
Solving Kepler's equation for arbitrary epoch and eccentricity
is a common problem in celestial mechanics. Below, I present
my implementation of such a solver, written in {\tt C}. It is
an extension of \citeauthor{1985CeMec..35..129S}'s (\citeyear{1985CeMec..35..129S}) work, and uses
ideas from \citet{1992fcm..book.....D} and \citet{1999CeMDA..74..275M}.

Given initial position and velocity and the time interval, the code
calculates the new position and the velocity of a particle with
the Hamiltonian
\begin{equation}
{\mathcal H} = \frac{p^2}{2m} - \frac{\mu}{r} -\frac{B_2}{r^2}\,.
\end{equation}

\section{Why a new routine}
There are of course codes already available for this purpose;
perhaps most notably the \texttt{SWIFT} 
routines\footnote{See: \url{http://www.boulder.swri.edu/~hal/swift.html}}.
They are heavily tested and {\em very} reliable. However, I wanted to
write my own routines, for a number of reasons.
\begin{itemize}
\item It was not straightforward for me to reach machine accuracy
with \swift\ routines. In particular, on my laptop (AMD, x86-64) with
GNU Fortran compiler, I explicitly had to use ``\verb|-mfpmath=387|''
flag. Unless this flag is used, the code generated uses the double
precision SSE instruction set. Unlike the 387 coprocessor instructions,
SSE instructions do not store the temporary results in 80 bit precision.
Seemingly this causes a small error (this was pointed out to me by
Patricia Verrier). However, the error is consistent, so it adds up linearly
in time, and decreasing the timestep makes things worse.\\
I probably did not need machine precision in the first place,
but it provides a peace of mind.
\item I wanted something compact and in \texttt{C}, so I could play
around with it more easily, parallelize the code, port it to GPUs etc.
\item I wanted to see if continued fractions led to faster code (The speed
improvement I obtained was negligible).
\item Continued fractions are intriguing on their own. The functions
we need to solve Kepler's equation have surprisingly simple continued
fraction expansions. Continued fractions allow infinite precision
arithmetic \citep{1972HAKMEM..239..101}. Well, I do not know what to do
about it at the moment, but it looks very interesting.
\end{itemize}
Please note that my experience with \swift\ routines is very limited.
It is possible that one can make a few modifications to increase
their accuracy while maintaining their reliability. I'd appreciate any
feedback on this matter. 

\section{Solving Kepler's equation}
For simplicity let us initially assume $B_2=0$. Then the orbit is a Keplerian conic
section. We start by calculating the quantities
\begin{equation}
r_0 = |\mathbf{r}_0|, \quad
v_0 = |\mathbf{v}_0|, \quad
\sigma_0 = \mathbf{r}_0 \mathbf{\cdot} \mathbf{v}_0, \quad
\beta = \frac{2\mu}{r_0} - v_0^2\,,
\end{equation}
where $\mathbf{r}_0$ and $\mathbf{v}_0$ are the
initial position and velocity vectors, respectively. If $\beta>0$,
we have an elliptical orbit and can calculate the period
\begin{equation}
P = 2\pi\mu \beta^{-3/2}\,,
\end{equation}
and the quantity
\begin{equation}
\delta U = n2\pi\beta^{-5/2},\quad 
n = \left\lfloor\frac{\delta t + P/2 -2\sigma_0/\beta}{P}\right\rfloor\,,
\end{equation}
where $\delta t$ is the given time interval. This will be useful if
$\delta t > P$, a situation that I do not encounter since I always
choose timesteps much shorter than the period.

For our independent variable, we choose the initial value $u=0$. The
main loop consists of the following steps (prime denotes differentiation
with respect to $u$):
\begin{gather}
q = \frac{\beta u^2}{1+\beta u^2} \,,\\
q' = \frac{2\beta u}{(1+\beta u^2)^2} \,,\\
q'' = \frac{2\beta}{(1+\beta u^2)^2} 
		  - \frac{8\beta^2 u^2}{(1+\beta u^2)^3} \,, \\
\tilde{U_0} = 1 - 2q \,,\\
\tilde{U_1} = 2(1-q)u \,,\\
U = \frac{16}{15} \tilde{U_1}^5 G_5(q) + \delta U \,,\\
U_0 = 2 \tilde{U_0}^2- 1 \,,\\
U_1 = 2 \tilde{U_0} \tilde{U_1} \,,\\
U_2 = 2 \tilde{U_1}^2 \,,\\
U_3 = \beta U + U_1 U_2 /3 \,,\\
r = r_0 U_0 + \sigma_0 U_1 + \mu U_2 \,,\\
r' =  4 (1-q) (\sigma_0 U_0 + (\mu-\beta r_0)U_1) \,,\\
r'' = -4 q' (\sigma_0 U_0 + (\mu-\beta r_0) U_1)
		       + 16 (1-q)^2  (-\beta \sigma_0 U_1 + (\mu-\beta r_0) U_0) \,,\\
\Delta t = r_0 U_1 + \sigma_0 U_2 + \mu U_3 \,,\\
f = \Delta t - \delta t \,,\\
f' = 4 (1-q) r \,,\\
f'' = 4 (r' (1-q) - r q')\,,\\
f''' = -8 r' q' - 4 r q'' + 4 (1-q) r'' \,,\\
\delta u_1  = -f/f' \,,\\
\delta u_2  = -f/(f' + \delta u_1 f''/2) \,,\\
\delta u_3  = -f/(f' + \delta u_2 f''/2 + \delta u_2^2 f'''/6) \,,\\
u = u+ \delta u_3\,.
\end{gather}
Once the increment $\delta u_3$ or $f$ is smaller than a preset
tolerance, we exit the loop.  The function $G_5(x)$ is a special case
of Gauss's hypergeometric function: $G_5(x) = {}_2F_1(5, 1; 7/2; x)$
\citep[][Ch. 15]{1972hmf..book.....A}.  The position and velocities
are calculated by
\begin{gather}
f = 1 - (\mu/ r_0) U_2 \,,\\
g = r_0 U_1 + \sigma_0 U_2   \,,\\
F = -\mu U_1/(r r_0) \,,\\
G = 1 - (\mu/r) U_2    \,,\\
\mathbf{r} = f \mathbf{r}_0 + g \mathbf{v}_0 \,,\\
\mathbf{v} = F \mathbf{r}_0 + G \mathbf{v}_0 \,.
\end{gather}
This method of solution is almost identical to \citeauthor{1985CeMec..35..129S}'s.
The only thing I did is to increase the
order of the iteration by using \citeauthor{1992fcm..book.....D}'s technique.
In the actual implementation, if $q>1/2$ during iteration, or if convergence
is not achieved after 12 iterations, we stop, and try to cover $\delta t$ in
two steps of $\delta t /2$.

When $B_2 \neq 0 $, we need two modifications.
In the following, I adopt the approach of \citet{1999CeMDA..74..275M};
see that paper for generating functions etc.
First note that in the plane of motion, we can transform to polar
coordinates $(r,\theta)$ and write the Hamiltonian as
\begin{equation}
\begin{split}
{\mathcal H} &= \frac{1}{2}\left(p_r^2 + \frac{p_\theta^2}{r^2}\right)
	       - \frac{\mu}{r} -\frac{B_2}{r^2} \\
& = \frac{1}{2}\left(p_r^2 + \frac{p_\theta^2 - 2B_2}{r^2}\right)
	       - \frac{\mu}{r} \\
& = \frac{1}{2}\left(p_r^2 + \frac{p_\psi^2}{r^2}\right)
	       - \frac{\mu}{r}\,,
\end{split}
\end{equation}
which is in Keplerian form. The first modification is
\begin{equation}
\beta = \frac{2\mu}{r_0} - v_0^2 + \frac{2 B_2}{r_0^2} \,.
\end{equation}
Furthermore, we cannot use $f$-$g$ formulation directly, so at the end of the loop,
we revert to a more general method, which is applicable for any motion with a radial force:
\begin{gather}
\mathbf{p}_\theta = \mathbf{r}_0 \times \mathbf{v}_0 \,,\\
p_\psi^2 = p_\theta^2 - 2 B_2 \,,\\
f = 1 - (\mu/ r_0) U_2 \,,\\
g = r_0 U_1 + \sigma_0 U_2   \,,\\
\xi = \frac{g}{r_0^2 f + \sigma_0 g}\,,\\
y_\psi = \xi  {\rm at}_1(p_\psi^2 \xi^2)\,,\\
\dot{r} = (\sigma_0 U_0 + (\mu - \beta r_0) U_1)/r \,,\\
\theta^2 = p_\theta^2 y_\psi^2 \,,\\
\mathbf{r} = \frac{r}{r_0} (c_0(\theta^2) \mathbf{r}_0 
		   + y_\psi c_1(\theta^2) \mathbf{p}_\theta \times \mathbf{r}_0)\,,\\
\mathbf{v} = \frac{\dot{r}}{r} \mathbf{r} + \frac{1}{r^2}\mathbf{p}_\theta\times\mathbf{r}\,.
\end{gather}
In this formulation $c_0(x^2)$ and $c_1(x^2)$ are Stumpff functions,
and ${\rm at}_1(x^2)$ is $\arctan(x)/x$.  All these functions, and the function
$G_5(x)$ mentioned earlier, have simple continued fraction expansions.
However, in my implementation, I only calculate $G_5(x)$ by continued fraction
expansion to keep things simple. An efficient way to do this is given by
\citep{1985CeMec..35..129S}. For calculating $c_0$ and $c_1$ by continued fractions,
probably the best way is to use the approach of \citet{cont_frac1987}. The
continued fraction for $\text{at}_1(x)$ goes back to Lambert (1770) and Lagrange (1776)
according to \citet[][Appendix II, formula 14]{1963cf..book.....O}:
\begin{equation}
\text{at}_1(x) = \frac{\arctan(\sqrt{x})}{\sqrt{x}} =
\cfrac{1}{1+\cfrac{1\cdot x}
	 {3+\cfrac{4\cdot x}
         {5+\cfrac{9\cdot x}
	 {7+\cfrac{16\cdot x}
	 {9+\cfrac{25\cdot x}{\ddots}}}}}}\,.
\end{equation}
%\section{Initial guesses}
%As with any iterative process, solving Kepler's equation benefits
%greatly from a good initial guess. A useful one is given by \citet[][eq. 6.9.38]{1992fcm..book.....D}
%as the root of the following cubic equation:
%\begin{equation}
%f_2(w) = \frac{1}{6}(\mu - \beta r_0) w^3 + \frac{1}{2} \sigma_0 w^2 + r_0 w -\delta t = 0 \, .
%\end{equation}
%He also gives a prescription for finding one root of this equation. Once we have an initial
%guess for $w$, we can transform this to an initial guess for $u$ by the following continued
%fraction \citep[][eq. 4.100]{1987aittmamof..book.....B}
%\begin{equation}
%u = \frac{U_1(w/4, \beta)}{U_0(w/4, \beta)} =
%\cfrac{w/4}{1-\cfrac{\beta(w/4)^2}
%	   {3-\cfrac{\beta(w/4)^2}
%           {5-\cfrac{\beta(w/4)^2}
%	   {7-\cfrac{\beta(w/4)^2} {\ddots}}}}}\,.
%\end{equation}
%\citeauthor{1992fcm..book.....D} recommends this for near parabolic orbits, but in
%my experience it is useful for all kinds of orbits; at least for small time increments
%which lead to small changes in eccentric anomaly. 
%
%With this initial guess, these routines clearly outperform the \texttt{SWIFT}
%routines. It is of course possible that the \texttt{SWIFT} routines would also benefit
%from this initial guess and the difference would disappear.

\bibliography{sol_kep}
\end{document}
